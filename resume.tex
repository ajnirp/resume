\documentclass[a4paper,9pt]{article}

\usepackage[margin=0.75in]{geometry}
\usepackage{inconsolata}
\usepackage{charter}
\usepackage[hidelinks,backref]{hyperref} % clickable links and citations with no green borders
\usepackage{amsmath}
\usepackage{listings} % add source code snippets
\usepackage{csquotes} % block quotes
\usepackage{color} % colors!
\usepackage[dvips]{graphicx}
\DeclareGraphicsExtensions{.png,.jpg}
\setlength{\parindent}{0pt}
\definecolor{mygreen}{rgb}{0,0.6,0}
\definecolor{mygray}{rgb}{0.5,0.5,0.5}
\definecolor{mymauve}{rgb}{0.58,0,0.82}

\lstset{ %
  backgroundcolor=\color{white},   % choose the background color; you must add \usepackage{color} or \usepackage{xcolor}
  basicstyle=\footnotesize\ttfamily,        % the size of the fonts that are used for the code
  breakatwhitespace=false,         % sets if automatic breaks should only happen at whitespace
  breaklines=true,                 % sets automatic line breaking
  captionpos=b,                    % sets the caption-position to bottom
  commentstyle=\color{mygreen},    % comment style
  deletekeywords={...},            % if you want to delete keywords from the given language
  escapeinside={\%*}{*)},          % if you want to add LaTeX within your code
  extendedchars=true,              % lets you use non-ASCII characters; for 8-bits encodings only, does not work with UTF-8
  frame=single,                    % adds a frame around the code
  keepspaces=true,                 % keeps spaces in text, useful for keeping indentation of code (possibly needs columns=flexible)
  keywordstyle=\color{blue},       % keyword style
  language=C++,                    % the language of the code
  morekeywords={*,...},            % if you want to add more keywords to the set
  numbers=none,                    % where to put the line-numbers; possible values are (none, left, right)
  numbersep=5pt,                   % how far the line-numbers are from the code
  numberstyle=\tiny\color{mygray}, % the style that is used for the line-numbers
  rulecolor=\color{black},         % if not set, the frame-color may be changed on line-breaks within not-black text (e.g. comments (green here))
  showspaces=false,                % show spaces everywhere adding particular underscores; it overrides 'showstringspaces'
  showstringspaces=false,          % underline spaces within strings only
  showtabs=false,                  % show tabs within strings adding particular underscores
  stepnumber=2,                    % the step between two line-numbers. If it's 1, each line will be numbered
  stringstyle=\color{mymauve},     % string literal style
  tabsize=2,                       % sets default tabsize to 2 spaces
  % title=\lstname                   % show the filename of files included with \lstinputlisting; also try caption instead of title
}

\hypersetup{
  colorlinks=true,
  linkcolor=blue,
  citecolor=blue,
  linktoc=page
}

\begin{document}

\vspace*{4.6cm}

% \Huge{\centerline{Rohan Prinja}}\vspace{31pt}
% font size help: https://engineering.purdue.edu/ECN/Support/KB/Docs/LaTeXChangingTheFont
\normalsize
\small

\Large{\textsc{Scholastic Achievements}}\vspace{1.5pt}
\hrule\vspace{0.25cm}
% \normalsize
\small
$\bullet$ Pursuing \textbf{honors} in Computer Science and Engineering\\
$\bullet$ \textbf{All India Rank 53} in IIT-JEE 2011 among 4.5 lakh candidates\\
$\bullet$ \textbf{All India Rank 51} in AIEEE 2011 among over 10 lakh candidates\\
$\bullet$ \textbf{All India Rank 7} in VITEEE 2011 (for admissions to \textbf{Vellore Institute of Technology})\\
$\bullet$ \textbf{All India Rank 130} in the 2011 entrance exam for the \textbf{Indian Insitute of Space Technology}\\
$\bullet$ Qualified for the \textbf{Indian National Physics Olympiad, 2010}\\
$\bullet$ Qualified for the \textbf{Indian National Mathematics Olympiad, 2011} (secured \textbf{rank 64} in the RMO)\\
$\bullet$ \textbf{All India Rank 1} in the 7\textsuperscript{\emph{th}} \textbf{National Cyber Olympiad (2005)} conducted by Science Olympiad Foundation\\

\Large{\textsc{Internship Projects}}\vspace{1.5pt}
\hrule\vspace{0.25cm}
% \normalsize
\small
$\bullet$ \textbf{Software Development Intern at Samsung Electronics, Korea}\hfill \textit{Summer 2014}\\
$\circ$ \textbf{Open source contributions}\\
-- Understood the \textbf{Chromium graphics stack} and patched Chromium's C\verb!++! 2D drawing library \textbf{Skia}\\
$\circ$ \textbf{Dodgebomb - Game development}\\
-- Developed a \textbf{WebGL} browser game using \textbf{three.js} that uses the W3C-standardized \textbf{Gamepad API}\\
-- Game runs as a webapp on \textbf{SBrowser} and \textbf{Chrome}. Controls are via the \textbf{Samsung Wireless Gamepad}. Will port to Tizen App Store (\textit{ongoing})\\
$\bullet$ \textbf{Software Development Intern at Chronus Software, India}\hfill \textit{Summer 2013}\\
$\circ$ \textbf{SSH Key Management}\\
-- Developed a CLI app in \textbf{Ruby} for privileged users (admins, ops team) to efficiently provision, rotate and revoke SSH key permissions for \textbf{AWS EC2} instances, using \textbf{AWS S3} as the keystore\\
$\circ$ \textbf{RTalk} [\href{https://github.com/wenderen/rtalk}{\texttt{github.com/wenderen/rtalk}}]\\
-- Anonymized chat app in \textbf{Rails} with emoticons, clickable links and an option to export chat history as a text file\\
-- Users create chat rooms and invite friends by sharing a randomized room URL\\
-- Room and chat logs are destroyed when everyone leaves\\

\Large{\textsc{Other Programming Projects}}\vspace{1.5pt}
\hrule\vspace{0.25cm}
% \normalsize
\small
\textbf{$\bullet$ Scrobbet} [\href{https://github.com/wenderen/scrobbet}{\texttt{github.com/wenderen/scrobbet}}] (\textit{ongoing}) \hfill \textit{August 2014}\\
-- Building a CLI/GUI app to send track metadata to \textbf{Last.fm}, a social-network/artist-database for music listeners\\
\textbf{$\bullet$ Time Glider} [Hackathon] [\href{https://github.com/wenderen/timeline-builder}{\texttt{github.com/wenderen/timeline-builder}}]\hfill \textit{August 2013}\\
-- Website that displays an interactive timeline of news articles related to a user's search term\\
-- Performed natural language processing in \textbf{Python NLTK} to summarize news articles. News article scraping and search indexing done in \textbf{Python} and \textbf{Apache Solr}. Frontend and interfacing done in \textbf{Sinatra}\\
-- Winner of the \textbf{second prize} at the \textbf{Yahoo! HackU 2013 IIT Bombay} event\\
\textbf{$\bullet$ Train Tracker} [Hackathon] \hfill \textit{July 2013}\\
-- Built a website that accepts a PNR number and returns \textbf{real-time} data about the upcoming train station (distance, time left, weather forecast, emergency numbers) to train travellers\\
-- Winner of the \textbf{Best Technical Hack} award at the \textbf{Yahoo! Hack India 2013 Hyderabad} event\\
-- Code written in \textbf{Ruby} using \textbf{Sinatra} as the framework.\\

\Large{\textsc{Research Experience}}\vspace{1.5pt}
\hrule\vspace{0.25cm}
% \normalsize
\small
\textbf{$\bullet$ Simulating Burning - Undergraduate Dissertation} (\textit{ongoing})\hfill \textit{Prof P Chaudhuri, Autumn 2014}\\
-- Physically modeling fire and incorporating burning, melting and residue formation for generic solid meshes\\
\textbf{$\bullet$ Bounding error-correcting codes} (\textit{ongoing})\hfill \textit{Prof A Kulkarni, Autumn 2014}\\
-- Deriving information-theoretic bounds on the cardinality of error-correcting codes for strings sent over permuting channels (tentative, topic may change)\\

\Large{\textsc{Teaching Experience}}\vspace{1.5pt}
\hrule\vspace{0.25cm}
% \normalsize
\small
$\bullet$ \textbf{Web and Coding Club, IIT Bombay}\hfill \textit{January 2013, January 2014}\\
-- Conducted two very well-received and popular \textbf{Python programming workshops} for beginners\\
$\bullet$ \textbf{Teaching Assistant, Computer Graphics course} (\textit{ongoing})\hfill \textit{Autumn 2014}\\
-- Developing teaching material for the course\\
-- Guiding students and helping them resolve their difficulties with the course\\

\Large{\textsc{Course Projects}}\vspace{1.5pt}
\hrule\vspace{0.25cm}
% \normalsize
\small
\textbf{$\bullet$ Rendering with PRman} [\href{https://wenderen.github.io/renderman/}{\texttt{wenderen.github.io/renderman/}}] \hfill \textit{Profs P Chaudhuri \& S Chandran, Spring 2014}\\
-- Wrote shaders and rendered raytraced scenes using Pixar's rendering software \textbf{Photorealistic RenderMan}\\
\textbf{$\bullet$ Musicbox Animation} [\href{https://github.com/wenderen/musicbox}{\texttt{github.com/wenderen/musicbox}}] \hfill \textit{Prof P Chaudhuri, Autumn 2013}\\
-- Modeled, textured, posed and animated a dancer in a musicbox. Built from scratch with OpenGL\\
\textbf{$\bullet$ MS Paint clone} [\href{https://github.com/wenderen/mydraw}{\texttt{github.com/wenderen/mydraw}}]\hfill \textit{Prof P Chaudhuri, Autumn 2013}\\
-- Basic MS Paint clone made with OpenGL with support for floodfill tool and checkered patterns\\
\textbf{$\bullet$ Elliptic Curve Cryptography Primitives} [\href{https://github.com/wenderen/ecc-asm}{\texttt{github.com/wenderen/ecc-asm}}]\hfill \textit{Prof B Menezes, Autumn 2013}\\
-- Implemented primitives for elliptic curve cryptography in MIPS Assembly\\
\textbf{$\bullet$ Service Bazaar} [\href{https://github.com/wenderen/service-bazaar}{\texttt{github.com/wenderen/service-bazaar}}]\hfill \textit{Prof U Bellur, Autumn 2013}\\
-- Built an Amazon/eBay/craigslist clone tailored towards services. Written using Rails\\
\textbf{$\bullet$ Fast Fourier Transform on FPGA}\hfill \textit{Prof A Gumasthe, Spring 2013}\\
-- Implemented eight-point radix-2 Decimation in Frequency FFT on Atlys Spartan 6 FPGA using VHDL.\\
% -- User gives input via DIP switches on the Spartan board. Output is displayed using the board LEDs.\\
\textbf{$\bullet$ Rube Goldberg Machine Simulator}\hfill \textit{Prof P Chaudhuri, Spring 2013}\\
-- Simulated a Rube Goldberg Machine in C\verb!++! using Box2D physics engine and OpenGL for rendering\\
\textbf{$\bullet$ FMoT: File Manager on Terminal}\hfill \textit{Prof V Apte, Fall 2012}\\
-- Wrote a terminal-based file manager in C\texttt{++} for Unix Systems supporting standard file management tasks\\
% tasks including \textit{cutting, copying, pasting, searching for files}\\
\textbf{$\bullet$ Chinese Checkers} [\href{https://github.com/wenderen/chinese-checkers}{\texttt{github.com/wenderen/chinese-checkers}}] \hfill \textit{Prof A Sanyal, Spring 2012}\\
-- Implemented heuristic game AIs for Chinese Checkers (Minimax and Alpha-beta pruning) with a GUI using Racket\\
% \textbf{$\bullet$ Statistical Analysis of Census Data}\hfill \textit{Spring 2012}\\
% -- Analysed census data using \textbf{Scilab} from two \textit{talukas} of Maharashtra regarding literacy levels, sex ratios and education levels, and used statistical techniques like regression analysis to infer trends in the \textit{talukas}\vspace{3pt}\\
% \textbf{$\bullet$ Monopoly Game}\hfill \textit{Fall 2011}\\
% -- Coded a complete program to simulate the popular board game \textbf{Monopoly} in \textbf{C++} using the  \textbf{EzWindow} graphics library\\
% -- Created \textit{supporting documentation}, weekly project reports, SRS, etc.\ for exposure to \textit{formal (team) programming} practices\\

\Large{\textsc{Programming and Technical Skills}}\vspace{1.5pt}
\hrule\vspace{0.25cm}
% \normalsize
\small
$\bullet$ \textbf{Knowledgeable about} C, C\verb!++!, OpenGL, JavaScript, WebGL, Ruby, Rails, Python\\
$\bullet$ \textbf{Basic familiarity with} SQL, Prolog, Assembly, Photorealistic Renderman, Amazon AWS (S3, IAM)\\

\Large{\textsc{Areas of Interest}}\vspace{1.5pt}
\hrule\vspace{0.25cm}
% \normalsize
\small
$\bullet$ \textbf{CS:} Graphics, Web Development, Functional Programming, Parallel Computing\\
$\bullet$ \textbf{Others:} Information Theory, Game Development\\

\Large{\textsc{Key Courses Undertaken}}\vspace{1.5pt}
\hrule\vspace{0.25cm}
% \normalsize
\small
\textbf{CS:} Digital Image Processing\textsuperscript{2}, Parallelizing Compilers\textsuperscript{2}, Parallel Computation\textsuperscript{2}, Advanced Computer Graphics, Computer Graphics, Operating Systems\textsuperscript{1}, Compilers\textsuperscript{1}, Artifical Intelligence\textsuperscript{1}, Networks\textsuperscript{1}, Databases\textsuperscript{1}, Computer Architecture\textsuperscript{1}, Theory of Computation, Logic Design\textsuperscript{1}, Design of Algorithms, Program Derivation, Software Systems, Data Structures\textsuperscript{1}, Discrete Structures, Abstractions and Paradigms\textsuperscript{1}\\
\textbf{Others:} Linear and Nonlinear Systems, Signals and Feedback Systems, Mathematical Structures for SysCon, Data Analysis \& Interpretation\hfill\textsuperscript{1}\emph{With associated lab course} \textsuperscript{2}\emph{Ongoing}\\

\Large{\textsc{Extra-curricular Activities}}\vspace{1.5pt}
\hrule\vspace{0.25cm}
% \normalsize
\small
$\bullet$ Passionate about foreign languages\\
-- Competent in \textbf{French} (studied as a second language for 7 years at the school level)\\
-- Completed one year courses in \textbf{German} (certified by \textbf{DAAD}) and \textbf{Chinese} (certified by \textbf{Beijing Jiaotong University})\\
$\bullet$ Built a self-controlled \textbf{wall-following robot} in a team of 3 for the \textsc{Wall-E} competition in \textbf{Techfest 2012}, which was selected among the top-32 teams, as well as a remote-controlled \textbf{racing robot} in a team of 3 for \textbf{Trackmania 2012}, which navigated an obstacle course\\
$\bullet$ Completed one year of \textbf{National Service Scheme} (2011-2012). Performed \textbf{‘shramdaan’} in rural areas of Maharashtra, visited NGOs and assisted in the building of a \textbf{tubewell}\\
$\bullet$ \textbf{Debating:} participated in and reached the finals of Sardar Patel Institute of Technology, Mumbai's debate competition in SPACE (annual cultural festival) 2012\\

\end{document}